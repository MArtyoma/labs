\chapter*{Вывод}
\addcontentsline{toc}{chapter}{Вывод}
\label{ch:Conclusion}

В ходе выполнения лабораторной работы были успешно решены две задачи по программированию на языке Java, направленные на закрепление навыков работы с базовыми конструкциями языка и алгоритмами.

Первая задача была посвящена поиску простых чисел. В процессе решения:

\begin{itemize}
  \item{Создан класс Primes с методом isPrime для проверки простоты числа}
  \item{Реализован эффективный алгоритм через тест Миллера-Рабина}
  \item{Организован вывод простых чисел}
  \item{Отработаны навыки работы с циклами и условными операторами}
\end{itemize}

Вторая задача касалась работы со строками и проверки палиндромов. В ходе её выполнения:

\begin{itemize}
\item{Разработан метод reverseString для переворота строки}
\item{Создан метод isPalindrome для проверки палиндрома}
\item{Отработаны навыки работы с методами length() и charAt()}
\item{Получен опыт обработки аргументов командной строки}
\end{itemize}

В результате выполнения работы достигнуты следующие цели:

\begin{itemize}
\item{Закреплены знания по работе с базовыми типами данных в Java}
\item{Получены практические навыки создания и использования методов}
\item{Отработаны алгоритмы проверки чисел и строк}
\item{Приобретен опыт разработки консольных приложений}
\end{itemize}

Обе программы успешно скомпилированы и протестированы, что подтверждает корректность реализованных алгоритмов и правильность их реализации. Полученные навыки являются фундаментальными для дальнейшего изучения программирования и разработки более сложных приложений.

\endinput



