\chapter*{Цель лабораторной работы}
\addcontentsline{toc}{chapter}{Цель лабораторной работы}
\label{ch:Goal}

Основная цель данной лабораторной работы заключается в практическом ознакомлении с базовыми возможностями языка программирования Java и формировании первичных навыков разработки простых приложений.

Конкретные задачи, которые необходимо решить в ходе выполнения работы:

\begin{itemize}
  \item{Изучить основы синтаксиса языка Java}
  \item{Освоить принципы работы с основными типами данных}
  \item{Получить практические навыки создания и компиляции Java-программ}
  \item{Научиться работать с переменными и операторами}
  \item{Познакомиться с базовыми конструкциями управления потоком выполнения программы}
  \item{Развить понимание принципов объектно-ориентированного программирования на начальном уровне}
  \item{Сформировать навыки отладки и тестирования простых Java-приложений}
\end{itemize}

Данная лабораторная работа направлена на создание прочной базы для дальнейшего изучения более сложных концепций и технологий Java-разработки.

\endinput

