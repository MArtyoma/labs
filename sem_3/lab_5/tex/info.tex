\chapter{Теоретические сведения}
% \label{ch:Task1}

\section{Класс String и его особенности}
Класс \texttt{String} в Java представляет собой неизменяемую последовательность символов Unicode. После создания объекта типа \texttt{String} его содержимое не может быть изменено. Любая операция, изменяющая строку, приводит к созданию нового объекта:

\begin{minted}{java}
String str = "Hello";
str += ", World!";
System.out.println(str); // Выведет "Hello, World!"
\end{minted}

Хотя переменная \texttt{str} кажется изменённой, на самом деле создаётся новая строка, а старая остаётся неизменной.

Преимущества неизменяемости:
\begin{itemize}
  \item Безопасность в многопоточной среде.
  \item Возможность кэширования хэш-кодов и интернирования строк.
\end{itemize}

Недостатки:
\begin{itemize}
  \item Высокое потребление памяти при множестве операций со строками.
\end{itemize}

Для частых изменений строк рекомендуется использовать \texttt{StringBuilder} или \texttt{StringBuffer}.

\section{Интернирование строк}
Интернирование — это механизм, при котором JVM сохраняет строковые литералы в пуле строк. Это позволяет избежать дублирования одинаковых строк:

\begin{minted}{java}
String s1 = "Hello";
String s2 = "Hello";
System.out.println(s1 == s2); // true
\end{minted}

Для строк, созданных через \texttt{new String()}, интернирование не происходит автоматически, но может быть выполнено вручную:

\begin{minted}{java}
String s3 = new String("World").intern();
String s4 = "World";
System.out.println(s3 == s4); // true
\end{minted}

\section{Unicode и длина строк}
Java использует кодировку UTF-16 для представления строк. Некоторые символы (например, эмодзи) занимают более одного 16-битного блока (суррогатная пара). Для корректного определения количества символов следует использовать метод \texttt{codePointCount()}:

\begin{minted}{java}
String str = "Привет!";
int lengthInCodePoints = str.length();
int lengthInCodeUnits = str.codePointCount(0, str.length());
System.out.println("Длина в code points: " + lengthInCodePoints);
System.out.println("Длина в code units: " + lengthInCodeUnits);
\end{minted}

\section{Регулярные выражения}
Регулярные выражения — мощный инструмент для поиска и обработки текста. В Java они реализованы через классы \texttt{Pattern} и \texttt{Matcher} из пакета \texttt{java.util.regex}.

Пример поиска слова:

\begin{minted}{java}
String text = "The quick brown fox jumps over the lazy dog";
Pattern pattern = Pattern.compile("fox");
Matcher matcher = pattern.matcher(text);
if (matcher.find()) {
    System.out.println("Match found!");
}
\end{minted}

\endinput
