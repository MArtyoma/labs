\chapter{Реализация}
% \label{ch:Task1}

\section{Задание 1. Среднее арифметическое элементов массива}

Реализуем программу, вычисляющую среднее арифметическое элементов массива с обработкой исключений: \texttt{ArrayIndexOutOfBoundsException} (выход за границы массива) и \texttt{NumberFormatException} (если элемент не является числом — в случае, если массив задан как строковый).

\begin{minted}{java}
public class ArrayAverage {
    public static void main(String[] args) {
        // "abc" вызовет исключение
        String[] arr = {"1", "2", "3", "4", "5", "abc"};
        int sum = 0;
        int count = 0;

        try {
            // Умышленная ошибка: i <= arr.length
            for (int i = 0; i <= arr.length; i++) {
                int num = Integer.parseInt(arr[i]);
                sum += num;
                count++;
            }
        } catch (ArrayIndexOutOfBoundsException e) {
            System.out.println("Ошибка: выход за границы массива.");
            e.printStackTrace();
        } catch (NumberFormatException e) {
            System.out.println("Ошибка: элемент массива не является числом.");
            e.printStackTrace();
        } finally {
            if (count > 0) {
                double average = (double) sum / count;
                System.out.println("Среднее арифметическое: " + average);
            } else {
                System.out.println("Невозможно вычислить среднее: нет корректных данных.");
            }
        }
    }
}
\end{minted}

\section{Задание 2. Копирование содержимого файла}

Реализуем программу копирования одного файла в другой с обработкой исключений, связанных с открытием/закрытием и чтением/записью файлов.

\begin{minted}{java}
import java.io.*;

public class FileCopy {
    public static void main(String[] args) {
        String sourceFile = "source.txt";
        String destFile = "destination.txt";

        FileInputStream fis = null;
        FileOutputStream fos = null;

        try {
            fis = new FileInputStream(sourceFile);
            fos = new FileOutputStream(destFile);

            int byteData;
            while ((byteData = fis.read()) != -1) {
                fos.write(byteData);
            }
            System.out.println("Файл успешно скопирован.");
        } catch (FileNotFoundException e) {
            System.out.println("Ошибка: файл не найден.");
            e.printStackTrace();
        } catch (IOException e) {
            System.out.println("Ошибка при чтении или записи файла.");
            e.printStackTrace();
        } finally {
            try {
                if (fis != null) fis.close();
                if (fos != null) fos.close();
            } catch (IOException e) {
                System.out.println("Ошибка при закрытии файлов.");
                e.printStackTrace();
            }
        }
    }
}
\end{minted}

\section{Задание 3 и 4. Пользовательское исключение для стека}

Создадим собственный класс исключения \texttt{CustomEmptyStackException} и класс \texttt{CustomStack}, имитирующий стек.

\begin{minted}{java}
// CustomEmptyStackException.java
public class CustomEmptyStackException extends Exception {
    public CustomEmptyStackException(String message) {
        super(message);
    }
}
\end{minted}

\begin{minted}{java}
// CustomStack.java
import java.util.ArrayList;
import java.io.FileWriter;
import java.io.IOException;
import java.time.LocalDateTime;

public class CustomStack<T> {
    private ArrayList<T> stack = new ArrayList<>();

    public void push(T item) {
        stack.add(item);
    }

    public T pop() throws CustomEmptyStackException {
        if (stack.isEmpty()) {
            logException("Попытка извлечь элемент из пустого стека");
            throw new CustomEmptyStackException("Стек пуст!");
        }
        return stack.remove(stack.size() - 1);
    }

    public boolean isEmpty() {
        return stack.isEmpty();
    }

    private void logException(String message) {
        try (FileWriter writer = new FileWriter("exceptions.log", true)) {
            writer.write(LocalDateTime.now() + " - " + message + "\n");
        } catch (IOException e) {
            System.err.println("Не удалось записать в лог: " + e.getMessage());
        }
    }
}
\end{minted}

\begin{minted}{java}
// Main.java — демонстрация работы
public class Main {
    public static void main(String[] args) {
        CustomStack<Integer> stack = new CustomStack<>();
        
        try {
            stack.pop(); // Вызовет исключение
        } catch (CustomEmptyStackException e) {
            System.out.println("Перехвачено исключение: " + e.getMessage());
        }
        
        stack.push(10);
        stack.push(20);
        try {
            System.out.println("Извлечено: " + stack.pop());
            System.out.println("Извлечено: " + stack.pop());
            stack.pop(); // Снова пустой стек
        } catch (CustomEmptyStackException e) {
            System.out.println("Перехвачено исключение: " + e.getMessage());
        }
    }
}
\end{minted}

При выполнении программы информация о каждом выброшенном исключении записывается в файл \texttt{exceptions.log}.

\endinput

