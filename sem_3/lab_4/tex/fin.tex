\chapter{Вывод}
% \label{ch:Task1}

В ходе выполнения лабораторной работы были изучены и применены на практике основные механизмы обработки исключений в Java. Были реализованы программы с использованием конструкций \texttt{try-catch-finally} для обработки стандартных исключений, таких как \texttt{ArrayIndexOutOfBoundsException}, \texttt{NumberFormatException} и \texttt{IOException}. Также было продемонстрировано создание пользовательского класса исключения \texttt{CustomEmptyStackException}, который позволяет более точно описывать ошибки в контексте конкретной задачи. Кроме того, реализована система логирования исключений в текстовый файл, что является важной практикой при разработке надёжных приложений. Полученные навыки позволяют эффективно управлять ошибками и повышать устойчивость программ к сбоям.

\href{https://github.com/MArtyoma/labs/tree/main/sem_3/lab_4}{Сслылка на git}

\endinput
