\chapter{Теоретические сведения}
% \label{ch:Task1}

В Java исключения делятся на две основные категории:

\begin{itemize}
  \item \textbf{Проверяемые исключения (checked exceptions)} — наследуются от класса \texttt{Exception} (но не от \texttt{RuntimeException}). Компилятор требует их обработки или объявления.
  \item \textbf{Непроверяемые исключения (unchecked exceptions)} — наследуются от \texttt{RuntimeException}. Не требуют обязательной обработки.
\end{itemize}

Основная конструкция обработки исключений выглядит следующим образом:

\begin{minted}{java}
try {
    // Код, который может вызвать исключение
} catch (ExceptionType1 e) {
    // Обработка исключения типа ExceptionType1
} catch (ExceptionType2 e) {
    // Обработка исключения типа ExceptionType2
} finally {
    // Код, который выполняется всегда, независимо от того,
    // возникло ли исключение или нет
}
\end{minted}

Ключевое слово \texttt{try} определяет блок кода, в котором может произойти исключение. Блоки \texttt{catch} перехватывают и обрабатывают конкретные типы исключений. Блок \texttt{finally} выполняется в любом случае — как при возникновении исключения, так и при его отсутствии. Это особенно полезно для освобождения ресурсов (например, закрытия файлов).

Кроме того, в Java можно создавать собственные классы исключений, наследуя их от \texttt{Exception} или его подклассов.

\endinput

