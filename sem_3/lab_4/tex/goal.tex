\chapter{Введение}
% \label{ch:Task1}

\section{Цель работы}
Изучить механизмы обработки исключений в языке программирования Java, научиться использовать конструкции \texttt{try-catch-finally}, создавать собственные классы исключений и применять их на практике при решении типовых задач.

\section{Введение}
Исключения в Java — это события, которые нарушают нормальный поток выполнения программы. Обработка исключений позволяет программе корректно реагировать на ошибки во время выполнения, предотвращая аварийное завершение и обеспечивая надёжность программного обеспечения. В данной лабораторной работе рассматриваются как стандартные механизмы обработки исключений, так и создание пользовательских исключений.

\endinput
