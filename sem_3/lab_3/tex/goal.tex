\chapter{Введение}
% \label{ch:Task1}

\section{Цель работы}

Разработка и реализация хэш-таблицы для эффективного хранения и обработки информации о продуктах, с освоением принципов работы хэш-структур данных и алгоритмов их обработки.

\subsection*{Конкретные задачи для достижения цели}

\begin{itemize}[label=$\bullet$]
    \item \textbf{Реализация структуры данных:} создание хэш-таблицы с использованием номера продуктах в качестве ключа
    \item \textbf{Разработка класса:} создание объекта для хранения информации о продукте (дата, список товаров, статус)
    \item \textbf{Программирование основных операций:}
        \begin{itemize}[label=$\circ$]
            \item вставка нового продукта в таблицу
            \item поиск продукта по номеру
            \item удаление продукта из таблицы
            \item изменение статуса существующего продукта
        \end{itemize}
    \item \textbf{Отработка практических навыков:}
        \begin{itemize}[label=$\circ$]
            \item применение хэш-функций
            \item обработка коллизий
            \item обновление продукта по ключу
        \end{itemize}
\end{itemize}
\section{Описание задачи}

Задание 1.

1. Создайте класс HashTable, который будет реализовывать
хэш-таблицу с помощью метода цепочек.

2. Реализуйте методы put(key, value), get(key) и remove(key),
которые добавляют, получают и удаляют пары «ключ-значение»
соответственно.

3. Добавьте методы size() и isEmpty(), которые возвращают
количество элементов в таблице и проверяют, пуста ли она.
Пример реализации метода put(key, value) представлен на ли-
стинге 3.3.

Листинг 3.3. Реализация метода put(key, value)
\begin{minted}{java}
public void put(K key, V value) {
  int index = hash(key);
  if (table[index] = = null) {
    table[index] = new LinkedList< Entry< K, V> > ();
  }
  for (Entry< K, V> entry : table[index]) {
    if (entry.getKey().equals(key)) {
      entry.setValue(value);
      return;
    }
  }
  table[index].add(new Entry< K, V> (key, value));
  size++;
}
\end{minted}

Реализация хэш-таблицы для учета продуктов на складе.
Ключом будет штрихкод товара, а значением — объект класса
Product, содержащий информацию о названии, цене и доступном
количестве. Необходимо реализовать операции вставки, поиска
и удаления продукта по штрихкоду

\endinput
