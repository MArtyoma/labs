\chapter{Реализация}
% \label{ch:Task1}

\section{Создание вспомогательного класса для хранения типа данных в хэш-таблице}

\inputminted{java}{../src/Entry.java}

Класс \textbf{Entry} представляет собой вспомогательный класс-обёртку для хранения пар ключ-значение в хэш-таблице. Данный класс является универсальным (generic) и может работать с любыми типами данных для ключа и значения.

\subsection{Назначение}

Класс предназначен для:
\begin{itemize}
    \item Хранения пары ключ-значение
    \item Использования в структурах данных на основе хэширования
    \item Обеспечения инкапсуляции данных
    \item Поддержки типизации через дженерики
\end{itemize}

\subsection{Структура класса}

\subsubsection{Параметры типа}
\begin{itemize}
    \item \textbf{K} - тип ключа (Key)
    \item \textbf{V} - тип значения (Value)
\end{itemize}

\subsubsection{Поля класса}
\begin{itemize}
    \item \textbf{key} - приватное поле для хранения ключа типа K
    \item \textbf{value} - приватное поле для хранения значения типа V
\end{itemize}

\subsubsection{Конструктор}

\begin{minted}{java}
public Entry(K key, V value)
\end{minted}
\begin{itemize}
    \item Принимает два параметра: ключ и значение
    \item Инициализирует соответствующие поля объекта
\end{itemize}

\subsubsection{Методы доступа}

\begin{minted}{java}
public K getKey()
\end{minted}
\begin{itemize}
    \item Возвращает сохранённый ключ
    \item Обеспечивает доступ только для чтения
\end{itemize}

\begin{minted}{java}
public V getValue()
\end{minted}
\begin{itemize}
    \item Возвращает сохранённое значение
    \item Обеспечивает доступ только для чтения
\end{itemize}

\begin{minted}{java}
public void setValue(V value)
\end{minted}
\begin{itemize}
    \item Позволяет изменить значение
    \item Сохраняет новое значение в поле value
\end{itemize}

\subsection{Особенности реализации}
\begin{itemize}
    \item Класс использует инкапсуляцию для защиты данных
    \item Все поля являются приватными
    \item Предоставляет контролируемый доступ через публичные методы
    \item Поддерживает работу с любыми типами данных благодаря дженерикам
\end{itemize}

\subsection{Пример использования}
\begin{minted}{java}
Entry<String, Integer> entry = new Entry<>("name", 42);
String key = entry.getKey(); // Получаем ключ
Integer value = entry.getValue(); // Получаем значение
entry.setValue(100); // Изменяем значение
\end{minted}

Данный класс является базовым строительным блоком для реализации более сложных структур данных, таких как HashMap, HashTable и других хэш-ориентированных коллекций.

\section{Создание класса для хранения данных о продукте}

% --- 

\inputminted{java}{../src/Product.java}

Класс \textbf{Product} представляет собой сущность для хранения информации о товаре. Класс содержит основные характеристики продукта: название, цену и количество.

\subsection{Назначение}

Класс предназначен для:
\begin{itemize}
    \item Хранения информации о товарах
    \item Обеспечения инкапсуляции данных
    \item Предоставления методов доступа к полям
    \item Форматированного вывода информации о товаре
\end{itemize}

\subsection{Структура класса}

\subsubsection{Поля класса}
\begin{itemize}
    \item \textbf{name} - приватное поле типа String для хранения названия товара
    \item \textbf{price} - приватное поле типа double для хранения цены
    \item \textbf{quantity} - приватное поле типа int для хранения количества товара
\end{itemize}

\subsubsection{Конструктор}
\begin{minted}{java}
public Product(String name, double price, int quantity)
\end{minted}
\begin{itemize}
    \item Инициализирует все поля объекта
    \item Принимает параметры для установки начальных значений
\end{itemize}

\subsubsection{Геттеры и сеттеры}

Для каждого поля определены методы доступа:

\begin{minted}{java}
public String getName()
public void setName(String name)
\end{minted}

\begin{minted}{java}
public double getPrice()
public void setPrice(double price)
\end{minted}

\begin{minted}{java}
public int getQuantity()
public void setQuantity(int quantity)
\end{minted}

\subsubsection{Метод toString}

\begin{minted}{java}
@Override
public String toString() {
    return "Product{name='" + name + "', price=" + price + ", quantity=" + quantity + "}";
}
\end{minted}

Предоставляет строковое представление объекта в формате:
\texttt{Product\{name='название', price=цена, quantity=количество\}}

\subsection{Особенности реализации}
\begin{itemize}
    \item Полная инкапсуляция данных
    \item Контроль доступа через геттеры и сеттеры
    \item Корректное строковое представление объекта
    \item Простота расширения функциональности
\end{itemize}

\subsection{Пример использования}
\begin{minted}{java}
Product product = new Product("Смартфон", 29999.99, 10);
String name = product.getName();
double price = product.getPrice();
int quantity = product.getQuantity();

System.out.println(product);
// Выведет: Product{name='Смартфон', price=29999.99, quantity=10}
\end{minted}

% --- 

\section{Создание класса HashTable}

\inputminted{java}{../src/HashTable.java}

Класс \textbf{HashTable} представляет собой реализацию хэш-таблицы с использованием обобщенных типов (generics). Хэш-таблица хранит пары ключ-значение и использует связанные списки для разрешения коллизий.

\subsection{Назначение}

Класс предназначен для:
\begin{itemize}
    \item Хранения пар ключ-значение
    \item Обеспечения быстрого доступа к данным по ключу
    \item Управления коллекцией элементов с помощью хэширования
    \item Разрешения коллизий через связанные списки
\end{itemize}

\subsection{Структура класса}

\subsubsection{Параметры типа}
\begin{itemize}
    \item \textbf{K} - тип ключа
    \item \textbf{V} - тип значения
\end{itemize}

\subsubsection{Поля класса}
\begin{itemize}
    \item \textbf{DEFAULT\_CAPACITY} - константа, задающая начальную емкость таблицы
    \item \textbf{table} - массив связанных списков для хранения элементов
    \item \textbf{size} - количество элементов в таблице
\end{itemize}

\subsubsection{Конструкторы}

\begin{minted}{java}
public HashTable()
\end{minted}
Создает таблицу с емкостью по умолчанию (16)

\begin{minted}{java}
public HashTable(int capacity)
\end{minted}
Создает таблицу с заданной емкостью

\subsubsection{Основные методы}

\paragraph{Хэширование}
\begin{minted}{java}
private int hash(K key)
\end{minted}
Вычисляет индекс для заданного ключа

\paragraph{Добавление элемента}
\begin{minted}{java}
public void put(K key, V value)
\end{minted}
Добавляет или обновляет элемент в таблице

\paragraph{Получение элемента}
\begin{minted}{java}
public V get(K key)
\end{minted}
Возвращает значение по заданному ключу

\paragraph{Удаление элемента}
\begin{minted}{java}
public V remove(K key)
\end{minted}
Удаляет элемент по заданному ключу и возвращает его значение

\subsubsection{Вспомогательные методы}
\begin{itemize}
    \item \textbf{size()} - возвращает количество элементов
    \item \textbf{isEmpty()} - проверяет пустоту таблицы
    \item \textbf{printTable()} - выводит содержимое таблицы
\end{itemize}

\subsection{Особенности реализации}
\begin{itemize}
    \item Использование обобщенных типов для гибкости
    \item Разрешение коллизий через связанные списки
    \item Инкапсуляция данных
    \item Контроль размера коллекции
    \item Возможность вывода содержимого таблицы
\end{itemize}

\subsection{Пример использования}
\begin{minted}{java}
HashTable<String, Integer> table = new HashTable<>();
table.put("key1", 10);
table.put("key2", 20);

Integer value = table.get("key1"); // Получаем значение
table.remove("key2"); // Удаляем элемент

table.printTable(); // Выводим содержимое таблицы
\end{minted}

\subsection{Применение}
Класс может использоваться в различных приложениях, где требуется:
\begin{itemize}
    \item Быстрый доступ к данным по ключу
    \textbf{Хранение пар ключ-значение}
    \item Реализация кэширования данных
    \item Создание ассоциативных массивов
    \item Обработка больших объемов данных
\end{itemize}

\endinput

