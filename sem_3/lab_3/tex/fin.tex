\chapter{Вывод}
% \label{ch:Task1}

В ходе выполнения лабораторной работы была успешно реализована структура данных \textbf{хэш-таблица} с использованием обобщенных типов (generics) и связанных списков для разрешения коллизий.

\section{Основные результаты работы}

1. Разработан вспомогательный класс \textbf{Entry}, который:
\begin{itemize}
    \item Обеспечивает хранение пар ключ-значение
    \item Реализует инкапсуляцию данных
    \item Предоставляет методы доступа к полям
\end{itemize}

2. Создан основной класс \textbf{HashTable}, который:
\begin{itemize}
    \item Реализует базовую функциональность хэш-таблицы
    \item Использует массив связанных списков для хранения данных
    \item Обеспечивает эффективное добавление, поиск и удаление элементов
\end{itemize}

3. В процессе реализации были достигнуты следующие цели:
\begin{itemize}
    \item Реализована \textbf{хэш-функция} для вычисления индекса элемента
    \item Реализовано \textbf{разрешение коллизий} через связанные списки
    \item Обеспечен \textbf{контроль размера} коллекции
    \item Реализована возможность \textbf{визуализации} содержимого таблицы
\end{itemize}

\section{Практическая значимость работы}

Практическая значимость работы заключается в следующем:
\begin{itemize}
    \item Получен опыт работы с обобщенными типами (generics)
    \item Изучены принципы работы хэш-таблиц
    \item Освоены методы разрешения коллизий
    \item Получена практика реализации сложных структур данных
\end{itemize}

\section{Решение поставленных задач}

В ходе работы были решены следующие задачи:
\begin{itemize}
    \item Создана эффективная структура хранения данных
    \item Реализованы основные операции с элементами
    \item Обеспечена безопасность типов через использование generics
    \item Реализована обработка особых случаев (пустые значения, коллизии)
\end{itemize}

\section{Анализ результатов}

Полученные результаты показывают, что разработанная хэш-таблица:
\begin{itemize}
    \item Обеспечивает быстрый доступ к данным
    \item Эффективно использует память
    \item Легко масштабируется
    \item Может быть использована в реальных приложениях
\end{itemize}

\section{Заключение}

Таким образом, цель лабораторной работы достигнута, основные задачи выполнены, а полученная реализация хэш-таблицы может служить основой для создания более сложных приложений, требующих эффективного хранения и обработки данных.

\endinput
