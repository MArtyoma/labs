\chapter{Теоретические сведения}
% \label{ch:Task1}


Хеш-таблицы являются одной из наиболее эффективных структур данных для быстрого поиска, добавления и удаления элементов. В Java они реализованы в классах \texttt{Hashtable} и \texttt{HashMap}.

\section{Основные концепции}

\subsection{Принцип работы}
Хеш-таблица представляет собой структуру данных, которая хранит пары ключ-значение. Для определения позиции элемента используется хеш-функция.

\subsection{Хеш-функция}
Хеш-функция преобразует ключ в числовой индекс массива. Формально:

\begin{equation}
f: K \rightarrow \{0, 1, ..., N-1\}
\end{equation}

где $K$ — множество ключей, $N$ — размер таблицы.

\section{Структура хеш-таблицы}

Хеш-таблица состоит из:
\begin{itemize}
    \item Массива бакетов (корзин)
    \item Механизма разрешения коллизий
    \item Хеш-функции
\end{itemize}

\section{Разрешение коллизий}
Коллизия возникает, когда разные ключи имеют одинаковый хеш-код. В Java используется метод цепочек:

\begin{minted}{java}
public class Hashtable<K, V> {
    private Entry<K, V>[] table;
    
    private static class Entry<K, V> {
        final K key;
        V value;
        Entry<K, V> next;
        
        Entry(K key, V value, Entry<K, V> next) {
            this.key = key;
            this.value = value;
            this.next = next;
        }
    }
}
\end{minted}

\section{Основные операции}

\subsection{Добавление элемента}
\begin{minted}{java}
public void put(K key, V value) {
    int hash = hash(key);
    int index = indexFor(hash, table.length);
    
    for (Entry<K, V> e = table[index]; e != null; e = e.next) {
        if (e.hash == hash && (e.key == key || key.equals(e.key))) {
            e.value = value;
            return;
        }
    }
    table[index] = new Entry<>(key, value, table[index]);
}
\end{minted}

\subsection{Поиск элемента}
\begin{minted}{java}
public V get(Object key) {
    int hash = hash(key);
    int index = indexFor(hash, table.length);
    
    for (Entry<K, V> e = table[index]; e != null; e = e.next) {
        if (e.hash == hash && (e.key == key || key.equals(e.key))) {
            return e.value;
        }
    }
    return null;
}
\end{minted}

\section{Характеристики производительности}
\begin{itemize}
    \item Средняя сложность операций: $O(1)$
    \item В худшем случае: $O(n)$ (при большом количестве коллизий)
\end{itemize}

\section{Заключение}
Хеш-таблицы являются мощным инструментом для работы с данными в Java, обеспечивая быстрый доступ и эффективное использование памяти при правильном использовании.

\endinput
