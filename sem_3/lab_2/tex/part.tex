\chapter{Задание}
\addcontentsline{toc}{chapter}{Задание}
\label{ch:Task1}

\begin{markdown}

**Описание программного кода лабораторной работы**

Представленный код демонстрирует реализацию **абстрактного класса** Bicycle, который служит основой для создания иерархии классов велосипедов. Рассмотрим основные компоненты и особенности реализации:

1. **Структура класса**
* Класс объявлен как абстрактный, что означает невозможность создания его экземпляров напрямую
* Используются **приватные поля** для хранения характеристик велосипеда:
  * model (модель)
  * yearOfManufacture (год выпуска)
  * hasBasket (наличие корзины)

2. **Система подсчета объектов**
* Реализован статический механизм подсчета созданных объектов через:
  * Статическую переменную instanceCount типа Map
  * Автоматическое увеличение счетчика при создании каждого объекта
  * Метод getInstanceCount для получения количества созданных объектов определенного класса

3. **Конструкторы класса**
* Предоставлены два варианта конструкторов:
  * Конструктор по умолчанию с инициализацией базовых значений
  * Параметризованный конструктор для установки конкретных характеристик

4. **Механизмы доступа**
* Реализованы **геттеры и сеттеры** для всех полей класса, обеспечивающие:
  * Контроль доступа к приватным полям
  * Поддержание инкапсуляции данных

5. **Абстрактный метод**
* Определен абстрактный метод getDetails(), который:
  * Должен быть реализован в классах-наследниках
  * Обеспечивает полиморфизм в иерархии классов

Данный код демонстрирует применение следующих принципов ООП:
* **Инкапсуляция** через использование приватных полей и публичных методов доступа
* **Наследование** через возможность создания производных классов
* **Абстракция** через определение базового функционала и абстрактного метода
* **Полиморфизм** через реализацию абстрактного метода в наследниках

Код служит основой для дальнейшей разработки иерархии классов велосипедов, демонстрируя корректную реализацию объектно-ориентированного подхода в программировании.

\end{markdown}

\section{Bicycle.java}
\inputminted{java}{../src/Bicycle.java}

\section{BMX.java}
\inputminted{java}{../src/BMX.java}

\section{MountainBike.java}
\inputminted{java}{../src/MountainBike.java}

\section{ChildrenBike.java}
\inputminted{java}{../src/ChildrenBike.java}

\endinput

