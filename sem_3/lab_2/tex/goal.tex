\chapter*{Цель лабораторной работы}
\addcontentsline{toc}{chapter}{Цель лабораторной работы}
\label{ch:Goal}


Закрепить теоретические знания и получить практические навыки проектирования и реализации объектно-ориентированной системы, включающей абстрактный класс и многоуровневое наследование. 

Конкретные задачи, которые необходимо решить:
\begin{itemize}
  \item{освоить принципы создания и использования абстрактных классов;}
  \item{научиться реализовывать многоуровневую иерархию наследования;}
  \item{получить навыки разработки классов с инкапсуляцией данных через геттеры и сеттеры;}
  \item{изучить механизмы работы конструкторов, включая конструктор по умолчанию;}
  \item{освоить применение статических переменных для реализации счётчика объектов;}
  \item{научиться реализовывать и демонстрировать все основные принципы объектно-ориентированного программирования (наследование, инкапсуляция, полиморфизм, абстракция);}
  \item{получить опыт работы с вводом и выводом информации об объектах.}
\end{itemize}

Ожидаемые результаты:
\begin{itemize}
  \item{создание функциональной объектно-ориентированной системы;}
  \item{демонстрация понимания принципов ООП на практике;}
  \item{формирование навыков проектирования иерархий классов;}
  \item{развитие умений реализации сложных программных конструкций.}
\end{itemize}

\endinput

