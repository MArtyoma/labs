\chapter*{Вывод}
\addcontentsline{toc}{chapter}{Вывод}
\label{ch:Conclusion}

\begin{markdown}

**Вывод по лабораторной работе**

В ходе выполнения лабораторной работы были успешно достигнуты поставленные цели и решены все поставленные задачи. 

**Основные результаты работы:**

1. Практическое освоение принципов объектно-ориентированного программирования:
* Создана иерархия классов с использованием абстрактного базового класса
* Реализованы механизмы наследования и инкапсуляции
* Продемонстрирован принцип полиморфизма через абстрактные методы
* Использована абстракция при проектировании базового класса

2. Полученные практические навыки:
* Разработка абстрактных классов и их наследников
* Создание многоуровневой системы наследования
* Реализация конструкторов различных типов
* Разработка механизмов доступа через геттеры и сеттеры
* Работа со статическими переменными и методами

3. Достигнутые результаты:
* Разработана функциональная система классов с учетом всех требований задания
* Реализован механизм подсчета созданных объектов
* Обеспечен корректный ввод и вывод информации об объектах
* Продемонстрирована работа всех компонентов системы

**Выводы:**

В результате выполнения лабораторной работы был создан работоспособный программный продукт, демонстрирующий основные принципы объектно-ориентированного программирования. Получены практические навыки проектирования и реализации иерархических структур классов, что является фундаментальным элементом современного программирования.

\end{markdown}

\endinput



