\chapter{Теоретические сведения}
% \label{ch:Task1}

\section{Основные интерфейсы коллекций}
В Java существует несколько основных интерфейсов коллекций:
\begin{itemize}
    \item \textbf{Collection} — базовый интерфейс коллекций
    \item \textbf{List} — упорядоченная коллекция с допущением дубликатов
    \item \textbf{Set} — коллекция уникальных элементов
    \item \textbf{Queue} — очередь (FIFO)
    \item \textbf{Deque} — двусторонняя очередь
    \item \textbf{Map} — ассоциативный массив (ключ-значение)
\end{itemize}

\section{Дженерики в Java}
Дженерики позволяют создавать классы, интерфейсы и методы, работающие с различными типами данных. Тип-параметр обозначается символом \texttt{<T>}. 

Ограничения дженериков:
\begin{itemize}
    \item \textbf{extends} — ограничение сверху
    \item \textbf{super} — ограничение снизу
\end{itemize}

\section{Стирание типов}
При компиляции информация о типах параметров стирается. Это приводит к ограничениям:
\begin{itemize}
    \item Невозможность создания массива дженериков
    \item Невозможность создания экземпляра параметризованного типа
    \item Отсутствие информации о типах на этапе выполнения
\end{itemize}



\endinput
