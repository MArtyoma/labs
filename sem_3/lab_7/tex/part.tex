\chapter{Реализация}

\section{Задание 1: Вычисление суммы элементов массива}

\textbf{Вариант 5: Использование CompletableFuture}

\inputminted{java}{../src/ArraySumCalculator.java}

\textbf{Пояснение:} В данной реализации массив разделяется на две части, для каждой из которых создается асинхронная задача с помощью CompletableFuture.supplyAsync(). Метод thenCombine() используется для комбинации результатов двух будущих вычислений.

\section{Задание 2: Поиск наибольшего элемента в матрице}

\textbf{Вариант 5: Использование CompletableFuture}

\inputminted{java}{../src/MatrixMaxFinder.java}

\textbf{Пояснение:} Для каждой строки матрицы создается асинхронная задача поиска максимального элемента. Затем с помощью цепочки вызовов thenCombine() происходит последовательное сравнение максимальных значений строк для нахождения глобального максимума.

\section{Задание 3: Моделирование работы грузчиков}

\textbf{Вариант 5: Использование CompletableFuture}

\inputminted{java}{../src/WarehouseSimulation.java}

\textbf{Пояснение:} В данной реализации три грузчика (представленные асинхронными задачами CompletableFuture) параллельно формируют партии товаров из общего склада. Каждая партия ограничена весом 150 кг. После формирования партии грузчики "доставляют" товары на другой склад, после чего процесс повторяется до полного опустошения склада.

\endinput

