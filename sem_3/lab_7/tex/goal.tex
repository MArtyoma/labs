\chapter{Введение}
% \label{ch:Task1}

\section{Цель работы}
Изучение принципов многопоточного программирования в Java, освоение механизмов создания и управления потоками, синхронизации доступа к общим ресурсам, а также практическое применение современных подходов к параллельным вычислениям с использованием CompletableFuture.

\section{Описание задачи}

\subsection{Постановка задачи}

В рамках данной лабораторной работы требуется разработать комплекс многопоточных приложений на языке Java, демонстрирующих различные аспекты параллельного программирования. Работа направлена на изучение теоретических основ многопоточности и приобретение практических навыков работы с современными механизмами параллельных вычислений.

\subsection{Технические требования}

\begin{itemize}
    \item Язык программирования: Java 8 или выше
    \item Использование современных API многопоточности
    \item Обеспечение потокобезопасности при работе с общими ресурсами
    \item Корректная обработка исключительных ситуаций в многопоточной среде
    \item Документирование кода и описание алгоритмов
\end{itemize}

\subsection{Задачи для реализации}

\subsubsection{Задача 1: Параллельное суммирование элементов массива}

\textbf{Цель:} Реализовать алгоритм параллельного вычисления суммы элементов большого массива с использованием современных средств многопоточности.

\textbf{Требования:}
\begin{itemize}
    \item Создать массив значительного размера (≥1000 элементов)
    \item Разделить вычисления между несколькими потоками
    \item Обеспечить корректное суммирование частичных результатов
    \item Реализовать проверку правильности вычислений
    \item Использовать вариант №5: \texttt{CompletableFuture}
\end{itemize}

\textbf{Ожидаемые результаты:}
\begin{itemize}
    \item Демонстрация ускорения вычислений за счет параллелизма
    \item Корректное суммирование элементов массива
    \item Пример использования \texttt{CompletableFuture} для асинхронных задач
\end{itemize}

\subsubsection{Задача 2: Поиск максимального элемента в матрице}

\textbf{Цель:} Разработать многопоточный алгоритм поиска наибольшего элемента в двумерной матрице.

\textbf{Требования:}
\begin{itemize}
    \item Создать матрицу значительного размера (≥100×100 элементов)
    \item Организовать параллельный поиск максимумов в строках/столбцах
    \item Обеспечить корректное сравнение частичных результатов
    \item Реализовать верификацию правильности работы алгоритма
    \item Использовать вариант №5: \texttt{CompletableFuture}
\end{itemize}

\textbf{Ожидаемые результаты:}
\begin{itemize}
    \item Эффективный параллельный поиск глобального максимума
    \item Демонстрация преимуществ многопоточного подхода
    \item Практическое применение цепочек \texttt{CompletableFuture}
\end{itemize}

\subsubsection{Задача 3: Моделирование логистической системы склада}

\textbf{Цель:} Создать имитационную модель работы грузчиков на складе с использованием продвинутых механизмов многопоточности.

\textbf{Требования:}
\begin{itemize}
    \item Реализовать классы: \texttt{Product}, \texttt{Warehouse}, \texttt{Loader}
    \item Организовать параллельную работу нескольких грузчиков
    \item Обеспечить ограничение веса переносимой партии (150 кг)
    \item Реализовать механизм синхронизации отправки партий
    \item Использовать вариант №5: \texttt{CompletableFuture}
\end{itemize}

\textbf{Бизнес-логика:}
\begin{itemize}
    \item Три грузчика работают параллельно
    \item Каждый грузчик формирует партию товаров ограниченного веса
    \item При достижении лимита веса партия "отправляется на другой склад"
    \item Процесс продолжается до полного опустошения склада
    \item Ведутся статистика работы каждого грузчика
\end{itemize}

\textbf{Ожидаемые результаты:}
\begin{itemize}
    \item Корректная работа системы в многопоточном режиме
    \item Отсутствие race condition и deadlock'ов
    \item Статистика работы системы
    \item Пример использования \texttt{CompletableFuture} для сложных асинхронных сценариев
\end{itemize}

\subsection{Критерии оценки}

\begin{itemize}
    \item \textbf{Корректность реализации} (40\%) — отсутствие ошибок синхронизации, race condition, deadlock'ов
    \item \textbf{Эффективность} (20\%) — оптимальное использование многопоточности, минимизация накладных расходов
    \item \textbf{Читаемость кода} (15\%) — хорошая структура, комментарии, соблюдение code style
    \item \textbf{Полнота реализации} (15\%) — выполнение всех требований задания
    \item \textbf{Документация} (10\%) — описание алгоритмов, выводы, ответы на вопросы
\end{itemize}

\subsection{Методология тестирования}

Для проверки корректности реализации предусмотрены:
\begin{itemize}
    \item Сравнение результатов с последовательными алгоритмами
    \item Многократный запуск для выявления race condition
    \item Тестирование на различных объемах данных
    \item Проверка статистической корректности в задаче 3
\end{itemize}

\subsection{Научная и практическая значимость}

Выполнение данной лабораторной работы позволяет:
\begin{itemize}
    \item Изучить фундаментальные принципы многопоточного программирования
    \item Освоить современные API Java для параллельных вычислений
    \item Приобрести навыки отладки многопоточных приложений
    \item Понять особенности проектирования потокобезопасных систем
    \item Подготовиться к решению реальных задач параллельного программирования
\end{itemize}

Работа демонстрирует эволюцию подходов к многопоточности в Java — от классических механизмов с использованием \texttt{Thread} и \texttt{Runnable} к современным асинхронным моделям с \texttt{CompletableFuture}, что отражает текущие тренды в разработке высокопроизводительных приложений.

\endinput

