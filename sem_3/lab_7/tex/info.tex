\chapter{Теоретические сведения}
% \label{ch:Task1}

\section{Понятие многопоточности}

Многопоточность — это способность программы выполнять несколько задач параллельно в рамках одного процесса. В Java многопоточность реализуется с помощью механизма потоков (threads), которые представляют собой легковесные процессы, разделяющие общее адресное пространство.

Каждый поток в Java имеет:
\begin{itemize}
    \item Собственный стек вызовов
    \item Программный счетчик
    \item Локальные переменные
    \item Доступ к общей памяти процесса
\end{itemize}

\section{Создание потоков}

В Java существует два основных способа создания потоков:

\section{Наследование от класса Thread}

\begin{minted}{java}
public class MyThread extends Thread {
    @Override
    public void run() {
        // Логика выполнения потока
        System.out.println("Выполнение потока: " + getName());
    }
}
\end{minted}

\section{Реализация интерфейса Runnable}

\begin{minted}{java}
public class MyRunnable implements Runnable {
    @Override
    public void run() {
        // Логика выполнения потока
        System.out.println("Выполнение потока: " + 
            Thread.currentThread().getName());
    }
}
\end{minted}

\section{Синхронизация потоков}

Для предотвращения состояний гонки (race condition) и обеспечения корректного доступа к общим ресурсам используются механизмы синхронизации:

\section{Ключевое слово synchronized}

\begin{minted}{java}
public class SynchronizedExample {
    private int counter = 0;
    
    public synchronized void increment() {
        counter++;
    }
}
\end{minted}

\section{Ключевое слово volatile}

\begin{minted}{java}
public class VolatileExample {
    private volatile boolean flag = false;
    
    public void setFlag(boolean value) {
        flag = value;
    }
}
\end{minted}

\section{CompletableFuture}

CompletableFuture — это современный подход к асинхронному программированию в Java, который предоставляет богатый API для композиции асинхронных операций.

Основные преимущества:
\begin{itemize}
    \item Цепочки асинхронных операций
    \item Комбинация нескольких будущих результатов
    \item Обработка исключений
    \item Ручное завершение операций
\end{itemize}

\endinput

