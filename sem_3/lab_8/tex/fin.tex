\chapter{Вывод}

В ходе выполнения лабораторной работы были достигнуты следующие результаты:

\begin{enumerate}
    \item Изучен механизм аннотаций в Java и создана пользовательская аннотация \texttt{@DataProcessor} для маркировки методов обработки данных с поддержкой приоритетов выполнения.
    
    \item Освоен функциональный подход к обработке данных с использованием Stream API, включая применение промежуточных операций (\texttt{filter}, \texttt{map}, \texttt{sorted}) и терминальных операций (\texttt{collect}, \texttt{forEach}).
    
    \item Реализована система многопоточной обработки данных с использованием классов пакета \texttt{java.util.concurrent}, в частности \texttt{ExecutorService} для управления пулами потоков и \texttt{CopyOnWriteArrayList} для потокобезопасного хранения данных.
    
    \item Разработана модульная архитектура приложения, позволяющая легко добавлять новые обработчики данных через механизм рефлексии и аннотаций.
    
    \item Продемонстрирована эффективность комбинирования различных технологий Java (аннотации, Stream API, многопоточность) для создания гибких и производительных приложений обработки данных.
\end{enumerate}

Полученное решение демонстрирует современный подход к разработке на Java, сочетающий объектно-ориентированное и функциональное программирование, и может быть расширено для решения более сложных задач обработки больших объемов данных.

\section*{Ответы на контрольные вопросы}

\begin{enumerate}
    \item \textbf{Что такое аннотации?} \\
    Аннотации — это форма метаданных, которые предоставляют дополнительную информацию о программе без изменения её семантики.
    
    \item \textbf{Для чего нужны аннотации?} \\
    Для маркировки кода, предоставления информации компилятору, обработки во время выполнения, генерации кода и документации.
    
    \item \textbf{Как создать собственную аннотацию?} \\
    Используя ключевое слово \texttt{@interface} с указанием политики удержания и целевых элементов.
    
    \item \textbf{Для чего можно использовать Stream API?} \\
    Для функциональной обработки коллекций данных, включая фильтрацию, преобразование, сортировку и агрегацию.
    
    \item \textbf{С помощью каких инструментов в Java можно обрабатывать данные параллельно?} \\
    Executor framework, ForkJoinPool, параллельные стримы, синхронизированные коллекции.
    
    \item \textbf{В чем разница между Collection и Stream?} \\
    Collection хранит данные, а Stream предоставляет операции для их обработки. Stream использует ленивое выполнение и не изменяет исходные данные.
    
    \item \textbf{Как создать Stream из коллекции? Массива? Отдельных элементов?} \\
    \texttt{collection.stream()}, \texttt{Arrays.stream(array)},\\
    \texttt{Stream.of(elements)}
    
    \item \textbf{Что такое Optional в Stream API?} \\
    Контейнер для значения, которое может быть null, позволяющий избежать \texttt{NullPointerException}.
    
    \item \textbf{Какие существуют терминальные операции Stream API?} \\
    \texttt{forEach}, \texttt{collect}, \texttt{reduce}, \texttt{count}, \texttt{findFirst}, \texttt{anyMatch}, \texttt{allMatch} и др.
\end{enumerate}

\href{https://github.com/MArtyoma/labs/tree/main/sem_3/lab_8}{Сслылка на git}

\endinput
