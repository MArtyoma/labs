\chapter{Реализация}
% \label{ch:Task1}

\section{Создание пользовательской аннотации}

Разработаем аннотацию \texttt{@DataProcessor} для маркировки методов обработки данных:

\inputminted{java}{../src/DataProcessor.java}

Данная аннотация обладает следующими характеристиками:
\begin{itemize}
    \item \texttt{@Retention(RetentionPolicy.RUNTIME)} - аннотация доступна во время выполнения
    \item \texttt{@Target(ElementType.METHOD)} - может применяться только к методам
    \item Содержит элементы \texttt{name} и \texttt{priority} со значениями по умолчанию
\end{itemize}

\section{Реализация класса DataManager}

Класс \texttt{DataManager} отвечает за координацию процесса обработки данных:

\inputminted{java}{../src/DataManager.java}

\section{Реализация обработчиков данных}

Создадим три различных обработчика данных, помеченных аннотацией \texttt{@DataProcessor}:

\inputminted{java}{../src/LengthFilter.java}

\inputminted{java}{../src/PrefixAdder.java}

\inputminted{java}{../src/UpperCaseTransformer.java}

\section{Тестирование приложения}

Создадим основной класс для тестирования функциональности:

\inputminted{java}{../src/Main.java}

\endinput


