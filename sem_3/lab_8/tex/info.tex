\chapter{Теоретические сведения}

\subsection{Аннотации в Java}

Аннотации представляют собой форму метаданных, которые могут быть добавлены в исходный код Java. Они не оказывают непосредственного влияния на семантику выполнения программы, но предоставляют дополнительную информацию, которая может быть использована на различных этапах жизненного цикла программы: во время компиляции, обработки инструментами разработки или в runtime-окружении.

Основные характеристики аннотаций:
\begin{itemize}
    \item Могут применяться к классам, методам, полям, параметрам и другим элементам программы
    \item Могут содержать элементы (атрибуты) со значениями по умолчанию
    \item Имеют политику удержания (RetentionPolicy), определяющую время жизни аннотации
\end{itemize}

\subsection{Stream API}

Stream API, введенное в Java 8, представляет собой абстракцию для последовательной обработки данных в функциональном стиле. Основные преимущества Stream API:

\begin{itemize}
    \item Декларативный стиль программирования
    \item Возможность параллельного выполнения операций
    \item Ленивое выполнение промежуточных операций
    \item Улучшенная читаемость кода
\end{itemize}

Архитектура Stream API состоит из трех основных компонентов:
\begin{enumerate}
    \item \textbf{Источник данных} - коллекции, массивы, генераторы
    \item \textbf{Промежуточные операции} - преобразуют элементы потока
    \item \textbf{Терминальные операции} - завершают обработку и производят результат
\end{enumerate}

\subsection{Многопоточность в Java}

Пакет \texttt{java.util.concurrent} предоставляет высокоуровневые средства для работы с многопоточностью, включая:
\begin{itemize}
    \item Executor framework для управления пулами потоков
    \item Синхронизированные коллекции (ConcurrentHashMap, \\
      CopyOnWriteArrayList)
    \item Примитивы синхронизации (Semaphore, ReentrantLock)
    \item Атомарные операции (AtomicInteger, AtomicReference)
\end{itemize}



\endinput
